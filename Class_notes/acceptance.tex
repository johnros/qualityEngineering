\chapter{Acceptance Sampling}
\chaptermark{Acceptance}

% AQL
% LTPD
% RQL
% LQL
% Type A and Type B RO curves.

We can improve quality (read- conformance to specification) by introducing an inspection stage in our process.
Clearly, a full inspection is time consuming. 
It may also be destructive (you don't want to re-package ice-cream after checking its texture \dots).
No-inspection may be appropriate if you don't particularly care about your brand, or if production has very high capability indices.
A reasonable, intermediate approach, is a partial random inspection, known as \emph{acceptance sampling}.
As the name suggests, in acceptance sampling, one samples, then checks, then accepts (or not).

Acceptance sampling can be seen as a control chart monitoring that triggers active intervention in the production. As such ,it is a crude type of \emph{engineering control} (Sec.~\ref{sec:terminology_statistical}).
The intervention is obvious. The monitoring is based on some continuous (variable) or discrete (attribute) of a sample of units from a \emph{batch}, \aka, a \emph{lot}.
Seen as a feedback control, it is not surprising that when designing an acceptance sampling scheme, we have similar decisions as when designing a control chart:
\begin{enumerate}
\item What is a batch? Just like choosing the sampling frequency in a Shewart chart. 
We would like homogenous batches, i.e., with low inner variability. A box, a shipment, a day's production, are typical batches. 
\item Within batch sampling scheme: just like rational grouping in Shewart chart. Typical approaches include \emph{single sampling plans}, \emph{double}, \emph{multiple}, and \emph{sequential sampling plans}.
\item How many units? Just like choosing the sample size in a Shewart chart.
\item Decision cutoff: Just like setting control limits in a Shewart chart. 
\end{enumerate}
We can readily see that the design of an acceptance sampling scheme is very similar to the design of a control chart. 
We may construct an full blown economical optimization problem to design the sampling, as we did in Section~\ref{sec:economical_considerations}. Just like control charts, however, it is more common to design sampling schemes using ``first-order'' power considerations. 
For this reason, the \emph{power function} will play a crucial role.

\section{Acceptance Sampling Terminology}
Adapted from \cite{natrella_nist/sematech_2010}.
\begin{description}
\item [LASP] A \emph{lot acceptance sampling plan}, as the name suggests. 
\item [AQL] The \emph{acceptable quality level}, or \emph{acceptable quality limit}, is the highest proportion of defects acceptable to the producer. 
\item [LTPD] The \emph{lot tolerance percent defective} is the highest proportion of defects acceptable to the consumer. Clearly, $AQL<LTPD$. LTPD is also known as \emph{rejectable quality level} (RQL), and \emph{ limiting quality level} (LQL). 
\end{description}

