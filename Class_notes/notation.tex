
% % % % % % Notation % % % % %


\chapter{Notation}
\label{apx:notation}

In this text we use the following notation conventions:
\begin{description}
\item[$x$] A column vector, or scalar, as implied by the text. 
\item[$:=$] An assignment, or definition. $A:=a$ means that $A$ is defined to be $a$. 
\item[$\prod_{i=1}^{n}$] The product operator: $\prod_{i=1}^{n} x_i:= x_1 \times \dots \times x_n$.
\item[$\coprod_{i=1}^{n}$] The coproduct operator: $\coprod_{i=1}^{n} x_i:= 1-(1-x_1) \times \dots \times (1-x_n)$.
\item[$\#\set{A}$] The count operator. Returns the number of elements of the set $A$. Also known as the \emph{cardinality}.
\item[$\Phi(t)$] The standard Gaussian CDF at $t$: $\Phi(t):= P(Z<t)$.
\item[$\phi(t)$] The standard Gaussian density at $t$: $\phi(t):= \frac{\partial}{\partial t}\Phi(t)$.
\item[$x'$] We use $'$ for the transpose operation. For a $1\times p$ row vector $x$, then $x'$ is a $p \times 1$ column vector.
\item[$\x_n \rightsquigarrow \dist$] Convergence in distribution: for large enough $n$, then $\x_n$ is distributed like $\dist$.
\item[$\conv$] The convolution operator: $f \conv g=(f \conv g)(t)=\int f(s) g(t-s) ds$.
\item[$f^{\conv n}$] The convolution power: $n$ convolutions of $f$ with itself.
\end{description}


