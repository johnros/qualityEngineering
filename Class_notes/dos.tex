\chapter{Design of Studies}
\chaptermark{DOS}
\label{ch:DOS}

\section{Types of Studies}


\section{Dealing with Systematic Errors}


\section{Dealing with Non-Systematic Errors}





\section{Sampling}
\subsection{Improving precision}
%TODO: stratification + covaraites. 
\subsection{Non-simple sampling}
Throughout this chapter we will be considering simple random samples, where all the population units have the same probability of being sampled. 
This is not always the case as illustrated in the next example.

\begin{example}[Respondent Driven Sampling]
	\label{ex:RDS}
	Consider the following chain-referral sampling scheme. 
	A Facebook questionnaire is passed by rewarding respondents some credit if they name other respondents.
	The probability of being sampled is thus not fixed for everyone, but rather a function of the number of your facebook friends.  
	Now imagine the questionnaire is meant to estimate the market share of product Y.
	It seems reasonable to assume that the more friends you have, the more likely you are to be sampled, and also, the more likely to be familiar with Y. 
	This sampling scheme will thus lead to an upward biased estimate of Y's popularity.
\end{example}

If the non-simple sampling scheme in Example~\ref{ex:RDS} is not accounted for at the analysis stage, we will clearly have biased popularity estimates. 
To deal with over-representation we will want to weight each sample by its probability of being sampled.
\begin{definition}[Horowitz Thompson Estimator]
	\label{def:horowitz-thompson}
	If each unit $i$ has probability $\pi_i$ of being sampled and measured response $y_i$, then an unbiased estimate of the population mean of $y$ is given by 
	\begin{align}
	\frac{\sum_{i \in S} y_i/\pi_i}{\sum_{i \in S} 1/\pi_i},
	\end{align}
	where $i \in S$ means that $i$ has actually been sampled. 
\end{definition}
You may verify that the Horowitz-Thompson estimator in Definition~\ref{def:horowitz-thompson} return the usual sample mean, if the sampling is simple, thus all $\pi_i$ are equal. 





\section{Bibliographic Notes}
