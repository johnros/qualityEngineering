\documentclass[12pt,a4paper]{report}


\usepackage[utf8]{inputenc}
\usepackage{amsmath}
\usepackage{amsfonts}
\usepackage{amssymb}
\usepackage{graphicx}
\usepackage{amsthm}
\usepackage{natbib}
\usepackage{algorithm}
\usepackage{algpseudocode}
\usepackage{framed}
\usepackage{todonotes}
\usepackage{tcolorbox}
\tcbuselibrary{breakable}




\usepackage{hyperref}
\AtBeginDocument{\let\textlabel\label}
\hypersetup{
    colorlinks=true,
    linkcolor=black,
    citecolor=black,
    filecolor=black,
    urlcolor=black,
}


\usepackage{marginnote}
\renewcommand*{\marginfont}{\scriptsize }

\usepackage{thmtools} % for lists of theorems


% Custom environments

\theoremstyle{plain}
\newtheorem{theorem}{Theorem}[section]
\newtheorem*{theorem*}{Theorem}
\newtheorem{lemma}{Lemma}[section]
\newtheorem*{lemma*}{Lemma}
\newtheorem{prop}{Proposition}[section]

\theoremstyle{definition}
\newtheorem{definition}{Definition}
\newtheorem{remark}{Remark}
\newtheorem{extra}{Extra Info}
\newtheorem{example}{Example}





% Custom commands

\newcommand{\naive}{na\"{\i}ve }
\newcommand{\Naive}{Na\"{\i}ve }
\newcommand{\andor}{and\textbackslash or }
\newcommand{\erdos}{Erd\H{o}s }
\newcommand{\renyi}{R\`enyi }


\newcommand{\al}{\alpha}
\newcommand{\be}{\beta}
\newcommand{\si}{\sigma}

\newcommand{\set}[1]{\{ #1 \}} % A set
\newcommand{\setII}[1]{\left\{ #1 \right\}} % A set
\newcommand{\rv}[1]{\mathbf{#1}} % A random variable
\newcommand{\x}{\rv x} % The random variable x 
\newcommand{\y}{\rv y} % The random variable x 
\newcommand{\T}{\rv t} % The random variable x 
\newcommand{\X}{\rv X} % The random variable x 
\newcommand{\Y}{\rv Y} % The random variable y
\newcommand{\expect}[1]{\mathbf{E}\left[ #1 \right]} % The expectation operator
\newcommand{\expectg}[2]{\mathbf{E}_{\rv{#1}}\left[ \rv{#2} \right]} % An expectation w.r.t. a particular random variable.
\newcommand{\expectn}[1]{\mathbb{E}\left[#1\right]} % The empirical expectation
\newcommand{\cov}[1]{\mathbf{Cov} \left[ #1 \right]} % The expectation operator
\newcommand{\var}[1]{\mathop{Var} \left[ #1 \right]} % The expectation operator
\newcommand{\covn}[1]{\mathbb{Cov} \left[ #1 \right]} % The expectation operator
\newcommand{\gauss}[1]{\mathcal{N}\left(#1\right)} % The gaussian distribution
\newcommand{\cdf}[2]{F_\rv{#1} (#2)} % The CDF function
\newcommand{\survive}[2]{S_\rv{#1} (#2)} % The survival function
\newcommand{\cdfn}[2]{\mathbb{F}_{#1}(#2)} % The empirical CDF function
\newcommand{\icdf}[2]{F_\rv{#1}^{-1} (#2)} % The invecrse CDF function
\newcommand{\icdfn}[2]{\mathbb{F}^{-1}_{#1}(#2)} % The inverse empirical CDF function
\newcommand{\pdf}{p} % The probability density function
\newcommand{\prob}[1]{P\left( #1 \right)} % the probability of an event
\newcommand{\dist}{P} % The proabaiblity distribution
\newcommand{\density}{p}
\newcommand{\entropy}{H} % entropy
\newcommand{\mutual}[2]{I\left(#1;#2\right)} % mutual information

\newcommand{\estim}[1]{\widehat{#1}} % An estimator
\newcommand{\estimII}[1]{\tilde{#1}} % Some other estimator

\newcommand{\norm}[1]{\Vert #1 \Vert} % The norm operator
\newcommand{\normII}[1]{\norm{#1}_2} % The norm operator
\newcommand{\normI}[1]{\norm{#1}_1} % The norm operator
\newcommand{\normF}[1]{\norm{#1}_{Frob}} % The Frobenius matrix norm
\newcommand{\ones}{\textbf{1}} % Vector of ones.
\newcommand{\lik}{\mathcal{L}} % The likelihood function
\newcommand{\loglik}{L} % The log likelihood function
\newcommand{\loss}{l} % A loss function
\newcommand{\lossII}{\prescript{}{2}{l}} % A loss function
\newcommand{\risk}{R} % The risk function
\newcommand{\riskn}{\mathbb{R}} % The empirical risk
\newcommand{\riskII}{\prescript{}{2}{R}} % The empirical risk
\newcommand{\risknII}{\prescript{}{2}{\mathbb{R}} } % The empirical risk
\newcommand{\noisen}{\mathbb{G}} % The empirical noise process
\newcommand{\deriv}[2]{\frac{\partial #1}{\partial #2}} % A derivative
\newcommand{\argmin}[2]{\textstyle{\mathop{argmin}_{#1}}\set{#2}} % The argmin operator
\newcommand{\argmax}[2]{\textstyle{\mathop{argmax}_{#1}}\set{#2}} % The argmin operator
\newcommand{\hyp}{f} % A hypothesis
\newcommand{\hypclass}{\mathcal{F}} % A hypothesis class
\newcommand{\hilbert}{\mathcal{H}}
\newcommand{\rkhs}{\hilbert_\kernel} % A hypothesis class
\newcommand{\normrkhs}[1]{\norm{#1}_{\rkhs}} % the RKHS function norm


\newcommand{\plane}{\mathbb{L}} % A hypoerplane
\newcommand{\categories}{\mathcal{G}} % The categories set.
\newcommand{\positive}[1]{\left[ #1 \right]_+} % The positive part function
\newcommand{\kernel}{\mathcal{K}} % A kernel function
\newcommand{\featureS}{\mathcal{X}} % The feature space
\newcommand{\outcomeS}{\mathcal{Y}} % The feature space
\newcommand{\indicator}[1]{I_{\set{#1}}} % The indicator function.
\newcommand{\reals}{\mathbb{R}} % the set of real numbers



\newcommand{\latent}{\rv{s}} % latent variables matrix
\newcommand{\latentn}{S} % latent variables matrix
\newcommand{\loadings}{A} % factor loadings matrix
\newcommand{\rotation}{R}  % rotation matrix
\newcommand{\similaritys}{\mathfrak{S}} % a similarity graph
\newcommand{\similarity}{s} % A similarity measure.
\newcommand{\dissimilarity}{d} % A dissimilarity measure.
\newcommand{\dissimilaritys}{\mathfrak{D}} % a dissimilarity graph
\newcommand{\scalar}[2]{\left< #1,#2 \right>} % a scalar product



\newcommand{\manifold}{\mathcal{M}} % A manifold.
\newcommand{\project}{\hookrightarrow} % The orthogonal projection operator.
\newcommand{\projectMat}{H} % A projection matrix.
\newcommand{\rank}{q} % A subspace rank.
\newcommand{\dimy}{K} % The dimension of the output.
\newcommand{\encode}{E} % a linear encoding matrix
\newcommand{\decode}{D} % a linear decoding matrix
\DeclareMathOperator{\Tr}{Tr}
\newcommand{\ensembleSize}{M} % Size of a hypothesis ensemble.
\newcommand{\ensembleInd}{m} % Index of a hypothesis in an ensemble.


\newcommand{\sample}{\mathcal{S}} % A data sample.
\newcommand{\test}{\risk(\hyp)} % The test error (risk)
\newcommand{\train}{\riskn(\hyp)} % The train error (empirical risk)
\newcommand{\insample}{\bar{\risk}(\hyp)} % The in-sample test error.
\newcommand{\EPE}{\risk(\hat{\hyp}_n)} % The out-of-sample test error.
\newcommand{\folds}{K} % Cross validation folds 
\newcommand{\fold}{k} % Index of a fold
\newcommand{\bootstraps}{B} % Bootstrap samples
\newcommand{\bootstrap}{{b^*}} % Index of a bootstrap replication


\newcommand{\rankings}{\mathcal{R}} % Rankings, for colaborative filtering.
\newcommand{\ranking}{\mathcal{R}} % Rankings, for colaborative filtering.
\newcommand{\KL}[2]{D_{KL}\left(#1 \Vert #2 \right)}
\newcommand{\ortho}{\mathbb{O}} % space of orthogonal matrices

\newcommand{\id}[6]{
	\begin{tabular}{|p{2cm}|p{2cm}|p{2cm}|p{2cm}|p{2cm}|p{2cm}|}
	\hline Task & Type & Input & Output & Concept & Remark \\ 
	\hline 
	\hline #1 & #2 & #3 & #4 & #5 & #6 \\ 
	\hline 
	\end{tabular} 
	\newline
	\newline
}

\newcommand{\union}{\cup}
\newcommand{\intersect}{\cap}
\newcommand{\supp}[1]{\mathop{support}(#1)}
\newcommand{\conf}[2]{\mathop{confidence}(#1 \Rightarrow #2)}
\newcommand{\lift}[2]{\mathop{lift}(#1 \Rightarrow #2)}
\newcommand{\convic}[2]{\mathop{conviction}(#1 \Rightarrow #2)}


\newcommand{\machine}[1]{\estim{\theta}_n^{(#1)}}
\newcommand{\minimizer}{\theta^*}
\newcommand{\generative}{\theta_0}
\newcommand{\parallelized}{\bar{\theta}_{N,m}}
\newcommand{\parallelizedII}{\mathring{\theta}_{N,m}}
\newcommand{\parallelizedIII}{\prescript{}{2}{\widehat{\theta}}_{N,m}}
\newcommand{\centralized}{\estim{\theta}_N}
\newcommand{\parallelKL}{\estim{\theta}_{KL}}
\newcommand{\penalize}{J}
\newcommand{\bigO}{\mathcal{O}}
\newcommand{\bigOprob}{\mathcal{O}_P}
\newcommand{\smallO}{o}
\newcommand{\smallOprob}{o_P}

\newcommand{\citeJR}[1]{\citeauthor{#1} \citep{#1}}
\newcommand{\citeJRfull}[1]{\citeauthor*{#1} \citep{#1}}
\newcommand{\error}{\mathcal{E}}

\newcommand{\M}{$M$}
\newcommand{\MII}{$\prescript{}{2}{M}$}

\newcommand{\biasSecond}[1]{B_2(#1)}
\newcommand{\MSESecond}[1]{M_2(#1)}

\newcommand{\rate}{r}

\newcommand{\emptyfigure}[1]{\missingfigure[figwidth=6cm]{#1}}


% % Time line
%\usepackage[paperwidth=210mm,%
%    paperheight=297mm,%
%    tmargin=7.5mm,%
%    rmargin=7.5mm,%
%    bmargin=7.5mm,%
%    lmargin=7.5mm,
%    vscale=1,%
%    hscale=1]{geometry}
%
%\usepackage[utf8]{inputenc}
%\usepackage[T1]{fontenc}

\usepackage{tikz}
\usetikzlibrary{arrows, calc, decorations.markings, positioning}


\makeatletter
\newenvironment{timeline}[6]{%
    % #1 is startyear
    % #2 is tlendyear
    % #3 is yearcolumnwidth
    % #4 is rulecolumnwidth
    % #5 is entrycolumnwidth
    % #6 is timelineheight

    \newcommand{\startyear}{#1}
    \newcommand{\tlendyear}{#2}

    \newcommand{\yearcolumnwidth}{#3}
    \newcommand{\rulecolumnwidth}{#4}
    \newcommand{\entrycolumnwidth}{#5}
    \newcommand{\timelineheight}{#6}

    \newcommand{\templength}{}

    \newcommand{\entrycounter}{0}

    % http://tex.stackexchange.com/questions/85528/checking-whether-or-not-a-node-has-been-previously-defined
    % http://tex.stackexchange.com/questions/37709/how-can-i-know-if-a-node-is-already-defined
    \long\def\ifnodedefined##1##2##3{%
        \@ifundefined{pgf@sh@ns@##1}{##3}{##2}%
    }

    \newcommand{\ifnodeundefined}[2]{%
        \ifnodedefined{##1}{}{##2}
    }

    \newcommand{\drawtimeline}{%
        \draw[timelinerule] (\yearcolumnwidth+5pt, 0pt) -- (\yearcolumnwidth+5pt, -\timelineheight);
        \draw (\yearcolumnwidth+0pt, -10pt) -- (\yearcolumnwidth+10pt, -10pt);
        \draw (\yearcolumnwidth+0pt, -\timelineheight+15pt) -- (\yearcolumnwidth+10pt, -\timelineheight+15pt);

        \pgfmathsetlengthmacro{\templength}{neg(add(multiply(subtract(\startyear, \startyear), divide(subtract(\timelineheight, 25), subtract(\tlendyear, \startyear))), 10))}
        \node[year] (year-\startyear) at (\yearcolumnwidth, \templength) {\startyear};

        \pgfmathsetlengthmacro{\templength}{neg(add(multiply(subtract(\tlendyear, \startyear), divide(subtract(\timelineheight, 25), subtract(\tlendyear, \startyear))), 10))}
        \node[year] (year-\tlendyear) at (\yearcolumnwidth, \templength) {\tlendyear};
    }

    \newcommand{\entry}[2]{%
        % #1 is the year
        % #2 is the entry text

        \pgfmathtruncatemacro{\lastentrycount}{\entrycounter}
        \pgfmathtruncatemacro{\entrycounter}{\entrycounter + 1}

        \ifdim \lastentrycount pt > 0 pt%
            \node[entry] (entry-\entrycounter) [below of=entry-\lastentrycount] {##2};
        \else%
            \pgfmathsetlengthmacro{\templength}{neg(add(multiply(subtract(\startyear, \startyear), divide(subtract(\timelineheight, 25), subtract(\tlendyear, \startyear))), 10))}
            \node[entry] (entry-\entrycounter) at (\yearcolumnwidth+\rulecolumnwidth+10pt, \templength) {##2};
        \fi

        \ifnodeundefined{year-##1}{%
            \pgfmathsetlengthmacro{\templength}{neg(add(multiply(subtract(##1, \startyear), divide(subtract(\timelineheight, 25), subtract(\tlendyear, \startyear))), 10))}
            \draw (\yearcolumnwidth+2.5pt, \templength) -- (\yearcolumnwidth+7.5pt, \templength);
            \node[year] (year-##1) at (\yearcolumnwidth, \templength) {##1};
        }

        \draw ($(year-##1.east)+(2.5pt, 0pt)$) -- ($(year-##1.east)+(7.5pt, 0pt)$) -- ($(entry-\entrycounter.west)-(5pt,0)$) -- (entry-\entrycounter.west);
    }

    \newcommand{\plainentry}[2]{% plainentry won't print date in the timeline
        % #1 is the year
        % #2 is the entry text

        \pgfmathtruncatemacro{\lastentrycount}{\entrycounter}
        \pgfmathtruncatemacro{\entrycounter}{\entrycounter + 1}

        \ifdim \lastentrycount pt > 0 pt%
            \node[entry] (entry-\entrycounter) [below of=entry-\lastentrycount] {##2};
        \else%
            \pgfmathsetlengthmacro{\templength}{neg(add(multiply(subtract(\startyear, \startyear), divide(subtract(\timelineheight, 25), subtract(\tlendyear, \startyear))), 10))}
            \node[entry] (entry-\entrycounter) at (\yearcolumnwidth+\rulecolumnwidth+10pt, \templength) {##2};
        \fi

        \ifnodeundefined{invisible-year-##1}{%
            \pgfmathsetlengthmacro{\templength}{neg(add(multiply(subtract(##1, \startyear), divide(subtract(\timelineheight, 25), subtract(\tlendyear, \startyear))), 10))}
            \draw (\yearcolumnwidth+2.5pt, \templength) -- (\yearcolumnwidth+7.5pt, \templength);
            \node[year] (invisible-year-##1) at (\yearcolumnwidth, \templength) {};
        }

        \draw ($(invisible-year-##1.east)+(2.5pt, 0pt)$) -- ($(invisible-year-##1.east)+(7.5pt, 0pt)$) -- ($(entry-\entrycounter.west)-(5pt,0)$) -- (entry-\entrycounter.west);
    }

    \begin{tikzpicture}
        \tikzstyle{entry} = [%
            align=left,%
            text width=\entrycolumnwidth,%
            node distance=10mm,%
            anchor=west]
        \tikzstyle{year} = [anchor=east]
        \tikzstyle{timelinerule} = [%
            draw,%
            decoration={markings, mark=at position 1 with {\arrow[scale=1.5]{latex'}}},%
            postaction={decorate},%
            shorten >=0.4pt]

        \drawtimeline
}
{
    \end{tikzpicture}
    \let\startyear\@undefined
    \let\tlendyear\@undefined
    \let\yearcolumnwidth\@undefined
    \let\rulecolumnwidth\@undefined
    \let\entrycolumnwidth\@undefined
    \let\timelineheight\@undefined
    \let\entrycounter\@undefined
    \let\ifnodedefined\@undefined
    \let\ifnodeundefined\@undefined
    \let\drawtimeline\@undefined
    \let\entry\@undefined
}
\makeatother
% % % % %

\newcommand{\R}{\textnormal{\sffamily\bfseries R }}

% Process capability notation
\newcommand{\targetValue}{T}% target value
\newcommand{\cp}{C_p}% c_p
\newcommand{\cpHat}{\hat{C}_p}% c_p
\newcommand{\ctqExpect}{\mu}
\newcommand{\pnc}{p_{NC}}
\newcommand{\cpu}{C_{pu}}
\newcommand{\cpl}{C_{pl}}
\newcommand{\cpk}{C_{pk}}
\newcommand{\cpm}{C_{pm}}
\newcommand{\cpq}{C_p(q)}
\newcommand{\pp}{P_{p}}
\newcommand{\ppk}{P_{pk}}

\newcommand{\barxChart}{$\bar{x}$-chart}
\newcommand{\sigmabar}{\sigma_{\bar{x}}}
\newcommand{\aka}{{a.k.a.\ }}
\newcommand{\Aka}{{A.k.a.\ }}
\newcommand{\rcode}[1]{\texttt{#1}}
\newcommand{\arm}{L}

\newcommand{\tsq}{$T^2$ }


\author{Jonathan Rosenblatt}
\title{Quality Engineering Class Notes (experimental)}

% % % % % % % % % % % % %


\begin{document}

\maketitle

\tableofcontents


%%%%%%%%% Algorithms %%%%%%%%%%%
%\newpage
%\listofalgorithms
%\addcontentsline{toc}{chapter}{List of Algorithms}

%\renewcommand{\listtheoremname}{List of Definitions}
%\listoftheorems[ignoreall,show={definition}]

%\renewcommand{\listtheoremname}{List of Examples}
%\listoftheorems[ignoreall,show={example}]



% % % Introduction % % % %

\chapter{Introduction}

Quality Engineering is the study and design of practices aimed improving the ``quality'' of production. 
Production is understood in a wide sense, and includes services as well.
Quality is understood in many senses. Here are several definitions compiled verbatim from \cite{montgomery_introduction_2007}  and \cite{wikipedia_quality_2015}:
\begin{enumerate}
\item Montgomery: ``Fitness to use''.
\item Montgomery: ``The reciprocal of variability'' (stability?) .
\item American Society for Quality:
A combination of quantitative and qualitative perspectives for which each person has his or her own definition; examples of which include, ``Meeting the requirements and expectations in service or product that were committed to'' and ``Pursuit of optimal solutions contributing to confirmed successes, fulfilling accountabilities.
 In technical usage, quality can have two meanings: 
 (a) The characteristics of a product or service that bear on its ability to satisfy stated or implied needs. 
 (b) A product or service free of deficiencies.''
\item Subir Chowdhury: 
``Quality combines people power and process power''.
\item Philip B. Crosby: 
``Conformance to requirements.''
\item  W. Edwards Deming:
``The efficient production of the quality that the market expects''.
\item W. Edwards Deming: 
``Costs go down and productivity goes up as improvement of quality is accomplished by better management of design, engineering, testing and by improvement of processes.''
\item Peter Drucker: 
``Quality in a product or service is not what the supplier puts in. It is what the customer gets out and is willing to pay for.''
\item Victor A. Elias: 
``Quality is the ability of performance, in each Theme of Performance, to enact a strategy.''
\item ISO 9000: 
``Degree to which a set of inherent characteristics fulfills requirements.'' 
\item Joseph M. Juran: 
``Fitness for use.''. 
\item Noriaki Kano and others, present a two-dimensional model of quality: ``must-be quality'' and ``attractive quality.'' The former is near to ``fitness for use'' and the latter is what the customer would love, but has not yet thought about. Supporters characterize this model more succinctly as: ``Products and services that meet or exceed customers' expectations.''
\item Robert Pirsig: ``The result of care.''
\item Six Sigma: ``Number of defects per million opportunities.''
\item Genichi Taguchi:
``Uniformity around a target value.''
\item Genichi Taguchi:
``The loss a product imposes on society after it is shipped.''
\item Gerald M. Weinberg: ``Value to some person''.
\item Jonathan D. Rosenblatt: ``The efficient fulfilment of a promise''.
\end{enumerate}



\begin{tcolorbox}[breakable]
\paragraph{Collecting ideas}
\begin{enumerate}
\item Quality is not only about production. 
\item Quality is the means, not the end.
\item Quality may deal with the \textbf{design} or with \textbf{conformance} to a given design. 
\end{enumerate}
\end{tcolorbox}


Almost all of the above definitions, may apply to different characteristics, we call \emph{dimensions of quality}. Following \cite{wikipedia_eight_2015} \marginnote{Dimensions of Quality}:
\begin{enumerate}
\item [Performance] Performance refers to a product's primary operating characteristics. This dimension of quality involves measurable attributes; brands can usually be ranked objectively on individual aspects of performance.
\item [{Features}] Features are additional characteristics that enhance the appeal of the product or service to the user.
\item [{Reliability}] Reliability is the likelihood that a product will not fail within a specific time period. This is a key element for users who need the product to work without fail.
\item [{Conformance}] Conformance is the precision with which the product or service meets the specified standards.
\item [{Durability}] Durability measures the length of a product’s life. When the product can be repaired, estimating durability is more complicated. The item will be used until it is no longer economical to operate it. This happens when the repair rate and the associated costs increase significantly.
\item [{Serviceability}] Serviceability is the speed with which the product can be put into service when it breaks down, as well as the competence and the behavior of the service person.
\item [{Aesthetics}] Aesthetics is the subjective dimension indicating the kind of response a user has to a product. It represents the individual’s personal preference.
\item [{Perceived Quality}] Perceived Quality is the quality attributed to a good or service based on indirect measures.
\end{enumerate}


\section{Terminology and Concepts}

\subsection{Basic Terminology}

\begin{enumerate}
\item [Quality Characteristics] A.k.a. \emph{Critical to Quality Characteristics} (CTQs). May be physical, sensory, or temporal properties of a process/product. Obviously related to the dimensions of quality. 

\item [Quality Engineering] ``The set of operational, managerial, and engineering activities
that a company uses to ensure that the quality characteristics of a product are at the nominal
or required levels and that the variability around these desired levels is minimum.'' \citep{montgomery_introduction_2007}

\item [Variables] Continuous measurements of some CTQ.

\item [Attributes] Discrete measurements of some CTQ.

\item [Target Value] The desired level of a particular CTQ. A.k.a. \emph{nominal} value. 

\item [Specifications] The set of target values of a process. 

\item [USL \& LSL] Largest and smallest allowable values of a CTQ.

\item [Non-conformity] A non conforming product is one that fails to meet the specification.

\item [Defect] A non-conformity that is serious enough to affect the use of the product.
\end{enumerate}




\subsection{Statistical Terminology}

\begin{enumerate}
\item [Exploratory Statistics]
\item [Causal Inference]
\item [Predictive Analytics]
\item [SPC] Statistical Process Control
\item [DOE] Design of experiments
\item [Acceptance Sampling] 
\item [Control Chart]
\item [Controllable Inputs]
\item [Uncontrollable Inputs]
\item [Factorial Design]
\item [Off-line quality control]
\item [On-line quality control] A.k.a. \emph{in-process} procedure. 
\item [Engineering control] A.k.a. \emph{automatic control}, or \emph{feedback control}.
\item [Ingoing Inspection]
\item [Outgoing Inspection]
\end{enumerate}




\section{Some History}

\begin{table}[H]
\footnotesize
\begin{timeline}{1875}{1948}{2cm}{2cm}{12cm}{12cm}
\entry{1875}{Frederick W. Taylor introduces ``Scientific Management''}
\entry{1900}{Henry Ford refines the assembly line to refine productivity and quality.}
\entry{1907}{AT\&T begins systematic inspections.}
\plainentry{1908}{W.S. Gosset publishes the t-test.}
\entry{1920}{AT\&T Bell labs establish a quality department.}
\plainentry{1920}{B. P. Dudding at General Electric in England uses statistical methods to control the quality of electric lamps}
\plainentry{1922}{R.A. Fisher inaugurates \emph{design of experiments}.}
\entry{1924}{W. A. Shewhart introduces the \emph{control chart} concept in a Bell Laboratories technical memorandum.}
\entry{1928}{Acceptance sampling refined by H. F. Dodge and H. G. Romig at Bell Labs.}
\entry{1933}{British textile and woolen industry and German chemical industry begin use of designed experiments
for product/process development.}
\entry{1946}{Deming is invited to Japan to help occupation forces in rebuilding Japanese industry.}
\plainentry{1948}{G. Taguchi begins study and application of experimental design.}
\end{timeline}
\caption{Adapted from \cite[Table 1.1]{montgomery_introduction_2007}.}
\end{table}


\begin{table}[H]
\footnotesize
\begin{timeline}{1951}{2000}{2cm}{2cm}{12cm}{12cm}
\entry{1951}{A. V. Feigenbaum publishes the first edition of his book, Total Quality Control.}
\plainentry{1951}{G. E. P. Box and K. B. Wilson publish fundamental work on designed experiments; focus is on chemical industry. Applications of designed experiments in the chemical industry grow steadily after this.}
\entry{1954}{Joseph M. Juran is invited by the Japanese to lecture on quality management and improvement.}
\entry{1960}{Courses in statistical quality control become widespread in industrial engineering academic programs.}
\entry{1987}{ISO publishes the first quality systems standard.}
\plainentry{1987}{Motorola’s six-sigma initiative begins.}
\entry{1997}{Motorola’s six-sigma approach spreads to other industries.}
\entry{2000}{ISO 9000:2000 standard is issued. Emphasis on supply-chain management and supplier quality. Expansion beyond the traditional industrial setting into financial services, health care, insurance.}
\end{timeline}
\caption{Adapted from \cite[Table 1.1]{montgomery_introduction_2007}.}
\end{table}






\section{Management Aspects of Improving Quality}

The founding fathers of QC have many dos-and-don'ts for managers.
See \citet[Sec 1.4]{montgomery_introduction_2007} for details. 
As usual, we collect recurring ideas:
\begin{tcolorbox}
\begin{enumerate}
\item The responsibility for quality rests with management. 
\item QC is not a one-time project, but an on-going process. It may advance continuously, or incrementally.
\item QC is (or should be) manifested in organizational structure, training, recruitment, incentives, knowledge management, to name a few.
\end{enumerate}
\end{tcolorbox}






% % % % Descriptive Statistics % % % % 
\chapter{Descriptive Statistics}

\section{Summary Statistics}

\subsection{Categorical Data}

\subsubsection{Univariate}
% counts/table
% entropy


\subsubsection{Bivariate}
% cross table
% cross entropy



\subsection{Continuous Data}

\subsubsection{Univariate}

\begin{enumerate}
\item[Location measures]
% mean, median, alpha-trimmed

\item[Scale measures]
% sd, mad, IQR, ranges

\item[Asymmetry measures]
% pearson, yule, wilcoxon

\end{enumerate}


\subsubsection{Bivariate}
\begin{enumerate}
\item[Correlation]
% mean, median, alpha-trimmed
\end{enumerate}










\section{Visualization}

\subsection{Categorical Data}

\subsubsection{Univariate}
% barplot

\subsubsection{Bivariate}
% mosaicplot


\subsection{Continuous Data}

\subsubsection{Univariate}
% stem and leaf
% histogram
% boxpolot


\subsubsection{Bivariate}
% scatter plot
% hexbinplot
% heatmap

\subsubsection{Multivariate Data}





% % % % Statistical inference % % % %
\chapter{Statistical Inference} 



\chapter{System Capability Analysis}




\chapter{Statistical Process Control}
\section{Univariate Control Charts}
\subsection{Control Charts for Variables}
\subsection{Control Charts for Attributes}



\section{Multivariate Control Charts}


\section{Non-Statistical Target Functions}



\chapter{Design of Experiments}


\chapter{Acceptance Sampling}


\chapter{Reliability}

% % % % % % Appendices % % % % % %
\newpage

\appendix





%%%%%%%%% Bibliography %%%%%%%%%%%
\newpage
\addcontentsline{toc}{chapter}{Bibliography}
\bibliographystyle{abbrvnat}
%\bibliography{QualityEngineering}
\bibliography{quality_clean}
\label{sec:bibliography}


\end{document}