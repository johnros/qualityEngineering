\chapter{Reliability Analysis}
\chaptermark{Reliability}

This chapter is adapted from \cite[Ch.8]{natrella_nist/sematech_2010} and German Rodriguez's Generalized-Linear-Models class notes\footnote{\url{http://data.princeton.edu/wws509/notes/c7.pdf}.}.

The attempt to define the difference between \emph{reliability} and \emph{quality} will certainly fail, since given the intentional ambiguity in our definition of quality (Chapter~\ref{sec:introduction}).
For our purposes, however, this terminological matter will not bother us, since we will simply define reliability analysis to be the analysis of the \emph{time} to \emph{failure}.
We will also assume that ``time'' and ``failure'' are well defined and agreed upon.

We intuitively understand ``more reliable'' to mean ``lasts longer''. 
We should also consider, however, the case of a product that is designed to fail after some time, thus forcing the consumer to buy a new one. 
Some may say that a major hi-tech company named after a fruit employs this practice. 
Be it true or not, I hope we can agree that good knowledge of your product's life expectancy is a desirable. 

Reliability analysis involves the study of a probabilistic property of our product- its survival.
Any probabilistic model will require calibration to reality via data. 
This chapter thus introduces both the probability calculus typically used for reliability analysis, and some statistical considerations involved when fitting these models.
But before the fun begins, we need some definitions and terminology.



% survival function
% hazard rate
% estimating survival
% probability calculus
% accelerated life models




\section{Terminology}
As usual we start with terminology.
It should be noted that reliability theory actually began with 19th century insurance companies, where is is known as \emph{survival analysis}.\marginnote{Survival Analysis}
In economics it is known as \emph{duration analysis}, and \emph{event history analysis} in sociology. 
This is why the terminology may sometimes seem ill-suited for services and products. 


\begin{definition}[CDF]
The cumulative distribution function (CDF) of a random variable $\T$ at a point $t$  is given by
\begin{align}
	F_\T(t):= P(\T<t).
\end{align}
\end{definition}


\begin{definition}[PDF]
The probability density function (PDF) of a continuous random variable $\T$ at a point $t$ is given by 
\begin{align}
	p_\T(t):= \frac{\partial}{\partial t}F_\T(t).
\end{align}

\end{definition}



\begin{definition}[Survival Function]
content...
\end{definition}



\begin{definition}[Hazard Function]
content...
\end{definition}





\begin{description}
\item [Death]
\item [Survival]
\item [Censored type-I]
\item [Censored type-I]
\item [Bathtub Function]
\end{description}






\section{Probabilistic Analysis}
[TODO]
% exponential, weibull, 
% competing risk
% series model
% parallel/redundant model
% r out of n model
% standby model
% complex systems




\section{Statistical Analysis}
[TODO]