\chapter{Statistical Process Control- SPC}
\label{sec:spc}

Statistical process control deals with the quantitative analysis of a ``process'', which may be a production line, a service, or any other repeated operation.
As such, it may be found in the Analyze, Improve, and Control stages of the DMAIC cycle.
The purpose of the SPC, in the terms coined by Shewart, is to seperate the variability in the process into \emph{assignable} causes of variation and \emph{chance} causes of variation.\marginnote{Causes of variation}
These are also known as \emph{special} and \emph{common} causes of variation, respectively. 
A process is said to be in \emph{statistical control} if all its variation is attributable to chance causes.
If this is not the case, then we will seek the assignable causes, remove then, and re-analyze.

All the previously mentioned statistical tools may be called upon for this analysis. 
In the context of process control, a subset of tools has gained the nick-name ``The Magnificent Seven''. These include:
\begin{description}
\item 1 1
\end{description}





%\begin{pgfpicture}
%    \pgftext{\pgfimage[width=0.6\linewidth, height=0.3\textheight]{}}
%\end{pgfpicture}


\section{A soft start. The $\bar{x}$ Chart}
% basic idea
% descion variables: sample size, intervals, statistic, limits, rational groupins. 
% what to do in case of alarm?
% extensions: probability limits, other statisics=, sample , adaptive parameters, other rules, non normality, variable sample size, moving windows
% considerations for setting these values.
% Setting limits: history, bootstrap, CLT
% multiplicity in control charts
% FDR controlling limits
% western electric rules. The type I error probability of the rule.
% nelson rules




\section{Other Control Statistic}
\subsection{$R$ Chart}
\subsection{$s$ Chart}
\subsection{$s^2$ Chart}
\subsection{Shewhart Individuals Control Chart}
\subsection{Three-way Chart}
\subsection{$p$ Chart}
\subsection{$np$ Chart}
\subsection{$c$ Chart}
\subsection{$u$ Chart}
\subsection{Time Series Model}
\subsection{Regression Control Chart}



\section{Running Window Charts}
\subsection{$EWMA$ Chart}
\subsection{$Cumsum$ Chart}


\section{Multivariate Control Charts}
% Wishart
% Srivastava Du
% PCA
% Higher criticism



\subsection{Non-Statistical Target Functions}





