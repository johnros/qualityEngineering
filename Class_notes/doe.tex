\chapter{Design of Experiments - DOE}

\section{Terminology}
Compiled from \cite{natrella_nist/sematech_2010}.
\begin{description}
\item [Alias] When the estimate of an effect also includes the influence of one or more other effects (usually high order interactions) the effects are said to be aliased.
\item [Balanced Design] An experimental design where all cells (i.e. treatment combinations) have the same number of observations.
\item [Blocking] A schedule for conducting treatment combinations in an experimental study such that any effects on the experimental results due to a known change in raw materials, operators, machines, etc., become concentrated in the levels of the blocking variable. Note: the reason for blocking is to isolate a systematic effect and prevent it from obscuring the main effects. Blocking is achieved by restricting randomization.
\item [Factor Encodings] Transforming the scale of measurement for a factor. 
Two level factor encodings include:
\begin{itemize}
\item Effect coding: where levels are encoded with $-1,1$ returning orthogonal design matrices for balanced designs.
\item Dummy coding: where levels are encoded with $0,1$ returning easily interpretable coefficients.
\end{itemize}

\item [Confounding]
\item [Control Group]
\item [Design]
\item [Design Matrix]
\item [Effect]
\item [Experimental Unit]
\item [Factors]
\item [Fixed Effect]
\item [Random Effect]
\item [Interaction]
\item [Lack of fit error]
\item [Pure Error]
\item [Orthogonality]
\item [Replication]
\item [Resolution]
\item [Response]
\item [Screening Design]
\item [Test Plan]
\item [Treatment]
\item [Treatment Group]
\item [Variance Components]
\end{description}

\section{Randomization}



\section{Full Factorial Designs}
\section{Fractional Factorial Designs}
\section{Response Surface Methods}
\section{Robust Parameter Design (RPD)}
\section{Latin-Square Design}
\section{Plackett–Burman Design}
\section{Computer Experiments}

