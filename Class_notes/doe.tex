\chapter{Design of Experiments - DOE}

This chapter is devoted to the matter of designing experiments.
A control chart may be seen as an on-line running experiment alerting us when things go sour. W
When designing a product (remember DFSS \ref{sec:dfss}), or one a control chart has signalled an alert, we will want to know what influences our production.
In our SPC terminology, we will want to know what are the \emph{causal} \emph{effect} of our \emph{controllable inputs}, or \emph{factors} influence our CTQ, or \emph{response}. 
The theory of discovering these effects is the theory of \emph{design of experiments} (DOE).

\begin{remark}[We do not discuss analysis]
In this text, we only discuss the design of the experiment, and not the analysis of the data.
This is a non-standard choice as DOE is typically presented alongside the \emph{analysis of variance} (ANOVA) which deals with the analysis of the data once this has been obtained.\marginnote{ANOVA}
We decouple the two since: 
(a) these are two different things, and 
(b) the language of ANOVA may be replaced, and has indeed been in many contexts, by the language of the \emph{linear models}.
There is a vast literature on analysis methods. 
If asked, this author may recommend \cite{hocking_analysis_1985}, which presents both the ANOVA terminology, and the linear models terminology.
That book, however, may be hard to come by, in which case, I also recommend \cite{greene_econometric_2003}, with a more econometric approach.
\end{remark}



\section{Terminology}
Compiled from \cite{natrella_nist/sematech_2010}.

\subsection{Major Terminology}

%Block. Group of homogeneous experimental units.
%Confounding. One or more effects that cannot unambiguously be attributed to
%a single factor or interaction.
%Covariate. An uncontrollable variable that influences the response but is
%unaffected by any other experimental factors.
%Design (layout). Complete specification of experimental test runs, including
%blocking, randomization, repeat tests, replication, and the assignment of
%factor–level combinations to experimental units.
%Effect. Change in the average response between two factor–level
%combinations or between two experimental conditions.
%Experimental region (factor space). All possible factor–level combinations
%for which experimentation is possible.
%Factor. A controllable experimental variable that is thought to influence the
%response.
%Homogeneous experimental units. Units that are as uniform as possible on
%all characteristics that could affect the response.
%Interaction. Existence of joint factor effects in which the effect of each factor
%depends on the levels of the other factors.
%Level. Specific value of a factor.
%Repeat tests. Two or more observations that have the same levels for all the
%factors.
%Replication. Repetition of an entire experiment or a portion of an experiment
%under two or more sets of conditions.
%Response. Outcome or result of an experiment.
%Test run. Single combination of factor levels that yields an observation on the
%response.
%Unit (item). Entity on which a measurement or an observation is made;
%sometimes refers to the actual measurement or observation.



\begin{description}
\item [Alias] When the estimate of an effect also includes the influence of one or more other effects (usually high order interactions) the effects are said to be aliased.
\item [Balanced Design] An experimental design where all cells (i.e. treatment combinations) have the same number of observations.
\item [Blocking] A schedule for conducting treatment combinations in an experimental study such that any effects on the experimental results due to a known change in raw materials, operators, machines, etc., become concentrated in the levels of the blocking variable. Note: the reason for blocking is to isolate a systematic effect and prevent it from obscuring the main effects. Blocking is achieved by restricting randomization.
\item [Factor Encodings] Transforming the scale of measurement for a factor. 
Two level factor encodings include:
\begin{itemize}
\item Effect coding: where levels are encoded with $-1,1$ returning orthogonal design matrices for balanced designs.
\item Dummy coding: where levels are encoded with $0,1$ returning easily interpretable coefficients.
\end{itemize}

\item [Confounding]
\item [Control Group]
\item [Covariate]
\item [Design]
\item [Design Matrix]
\item [Effect]
\item [Experimental Unit]
\item [Factors]
\item [Fixed Effect]
\item [Random Effect]
\item [Interaction]
\item [Lack of fit error]
\item [Pure Error]
\item [Orthogonality]
\item [Replication]
\item [Resolution]
\item [Response]
\item [Screening Design]
\item [Test Plan]
\item [Treatment]
\item [Treatment Group]
\item [Variance Components]
\end{description}



\subsection{Minor Terminology}


\section{Randomization}
% missing factors, added to variance, not bias.
% Other consideration for design
% causal inference



\section{Pre-Experiment}





\section{Fully Randomized Designs}
% complete factorial
% importance of factorial when optimizing a process
% Regression and ANOVA terminology (reference: the analysis of linear models)
% notation A,B, (1),a,b,ab
% interaction plot
% a single replication design

\subsection{Full Factorial}

\subsection{Fractional Factorial}

\subsubsection{Confounding of Effects}

\subsubsection{Design Resolution}


\subsection{Screening Experiments}

\section{Random Effects}

\subsection{Randomized Block Designs}

\subsubsection{Complete Block Designs}


\subsubsection{Latin Square Design}

\subsubsection{Latin Square Design}

\subsubsection{Cross Over Design}

\subsubsection{Incomplete Block Designs}


\subsection{Random Effects Facotor}

\subsection{Nested Designs}


\subsubsection{Nested and Crossed Designs}

\subsubsection{Split Plot Designs}




\subsection{Process Improvement Designs}

\subsubsection{Taguchi's Robust Design Approach}



\section{Continuous Factors}



\subsection{Response Surface Methdology}

\subsubsection{Central Composite Design}

\subsubsection{Box-Behnken Design}




\section{Sequential Designs}

\subsection{Po}



\section{Adaptive Designs}



\section{Computer Experiments}

