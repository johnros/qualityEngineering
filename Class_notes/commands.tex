% Custom environments

\theoremstyle{plain}
\newtheorem{theorem}{Theorem}[section]
\newtheorem*{theorem*}{Theorem}
\newtheorem{lemma}{Lemma}[section]
\newtheorem*{lemma*}{Lemma}
\newtheorem{prop}{Proposition}[section]

\theoremstyle{definition}
\newtheorem{definition}{Definition}
\newtheorem{remark}{Remark}
\newtheorem{extra}{Extra Info}
\newtheorem{example}{Example}





% Custom commands

\newcommand{\naive}{na\"{\i}ve }
\newcommand{\Naive}{Na\"{\i}ve }
\newcommand{\andor}{and\textbackslash or }
\newcommand{\erdos}{Erd\H{o}s }
\newcommand{\renyi}{R\`enyi }


\newcommand{\al}{\alpha}
\newcommand{\be}{\beta}
\newcommand{\si}{\sigma}

\newcommand{\set}[1]{\{ #1 \}} % A set
\newcommand{\setII}[1]{\left\{ #1 \right\}} % A set
\newcommand{\rv}[1]{\mathbf{#1}} % A random variable
\newcommand{\x}{\rv x} % The random variable x 
\newcommand{\y}{\rv y} % The random variable x 
\newcommand{\U}{\rv u} % The random variable x 
\newcommand{\T}{\rv t} % The random variable x 
\newcommand{\X}{\rv X} % The random variable x 
\newcommand{\Y}{\rv Y} % The random variable y
\newcommand{\expect}[1]{\mathbf{E}\left[ #1 \right]} % The expectation operator
\newcommand{\expectg}[2]{\mathbf{E}_{\rv{#1}}\left[ \rv{#2} \right]} % An expectation w.r.t. a particular random variable.
\newcommand{\expectn}[1]{\mathbb{E}\left[#1\right]} % The empirical expectation
\newcommand{\cov}[1]{\mathbf{Cov} \left[ #1 \right]} % The expectation operator
\newcommand{\var}[1]{\mathop{Var} \left[ #1 \right]} % The expectation operator
\newcommand{\covn}[1]{\mathbb{Cov} \left[ #1 \right]} % The expectation operator
\newcommand{\gauss}[1]{\mathcal{N}\left(#1\right)} % The gaussian distribution
\newcommand{\cdf}[2]{F_{#1} (#2)} % The CDF function
\newcommand{\survive}[2]{S_{#1} (#2)} % The survival function
\newcommand{\hazard}[2]{\lambda_{#1} (#2)} % The survival function
\newcommand{\cuhazard}[2]{\Lambda_{#1} (#2)} % The survival function
\newcommand{\cdfn}[2]{\mathbb{F}_{#1}(#2)} % The empirical CDF function
\newcommand{\icdf}[2]{F_\rv{#1}^{-1} (#2)} % The invecrse CDF function
\newcommand{\icdfn}[2]{\mathbb{F}^{-1}_{#1}(#2)} % The inverse empirical CDF function
\newcommand{\pdf}[2]{p_{#1} (#2)} % The CDF function
\newcommand{\prob}[1]{P\left( #1 \right)} % the probability of an event
\newcommand{\dist}{P} % The proabaiblity distribution
\newcommand{\density}{p}
\newcommand{\entropy}{H} % entropy
\newcommand{\mutual}[2]{I\left(#1;#2\right)} % mutual information

\newcommand{\estim}[1]{\widehat{#1}} % An estimator
\newcommand{\estimII}[1]{\tilde{#1}} % Some other estimator

\newcommand{\norm}[1]{\Vert #1 \Vert} % The norm operator
\newcommand{\normII}[1]{\norm{#1}_2} % The norm operator
\newcommand{\normI}[1]{\norm{#1}_1} % The norm operator
\newcommand{\normF}[1]{\norm{#1}_{Frob}} % The Frobenius matrix norm
\newcommand{\ones}{\textbf{1}} % Vector of ones.
\newcommand{\lik}{\mathcal{L}} % The likelihood function
\newcommand{\loglik}{L} % The log likelihood function
\newcommand{\loss}{l} % A loss function
\newcommand{\lossII}{\prescript{}{2}{l}} % A loss function
\newcommand{\risk}{R} % The risk function
\newcommand{\riskn}{\mathbb{R}} % The empirical risk
\newcommand{\riskII}{\prescript{}{2}{R}} % The empirical risk
\newcommand{\risknII}{\prescript{}{2}{\mathbb{R}} } % The empirical risk
\newcommand{\noisen}{\mathbb{G}} % The empirical noise process
\newcommand{\deriv}[2]{\frac{\partial #1}{\partial #2}} % A derivative
\newcommand{\argmin}[2]{\textstyle{\mathop{argmin}_{#1}}\set{#2}} % The argmin operator
\newcommand{\argmax}[2]{\textstyle{\mathop{argmax}_{#1}}\set{#2}} % The argmin operator
\newcommand{\hyp}{f} % A hypothesis
\newcommand{\hypclass}{\mathcal{F}} % A hypothesis class
\newcommand{\hilbert}{\mathcal{H}}
\newcommand{\rkhs}{\hilbert_\kernel} % A hypothesis class
\newcommand{\normrkhs}[1]{\norm{#1}_{\rkhs}} % the RKHS function norm


\newcommand{\plane}{\mathbb{L}} % A hypoerplane
\newcommand{\categories}{\mathcal{G}} % The categories set.
\newcommand{\positive}[1]{\left[ #1 \right]_+} % The positive part function
\newcommand{\kernel}{\mathcal{K}} % A kernel function
\newcommand{\featureS}{\mathcal{X}} % The feature space
\newcommand{\outcomeS}{\mathcal{Y}} % The feature space
\newcommand{\indicator}[1]{I_{\set{#1}}} % The indicator function.
\newcommand{\reals}{\mathbb{R}} % the set of real numbers



\newcommand{\latent}{\rv{s}} % latent variables matrix
\newcommand{\latentn}{S} % latent variables matrix
\newcommand{\loadings}{A} % factor loadings matrix
\newcommand{\rotation}{R}  % rotation matrix
\newcommand{\similaritys}{\mathfrak{S}} % a similarity graph
\newcommand{\similarity}{s} % A similarity measure.
\newcommand{\dissimilarity}{d} % A dissimilarity measure.
\newcommand{\dissimilaritys}{\mathfrak{D}} % a dissimilarity graph
\newcommand{\scalar}[2]{\left< #1,#2 \right>} % a scalar product



\newcommand{\manifold}{\mathcal{M}} % A manifold.
\newcommand{\project}{\hookrightarrow} % The orthogonal projection operator.
\newcommand{\projectMat}{H} % A projection matrix.
\newcommand{\rank}{q} % A subspace rank.
\newcommand{\dimy}{K} % The dimension of the output.
\newcommand{\encode}{E} % a linear encoding matrix
\newcommand{\decode}{D} % a linear decoding matrix
\DeclareMathOperator{\Tr}{Tr}
\newcommand{\ensembleSize}{M} % Size of a hypothesis ensemble.
\newcommand{\ensembleInd}{m} % Index of a hypothesis in an ensemble.


\newcommand{\sample}{\mathcal{S}} % A data sample.
\newcommand{\test}{\risk(\hyp)} % The test error (risk)
\newcommand{\train}{\riskn(\hyp)} % The train error (empirical risk)
\newcommand{\insample}{\bar{\risk}(\hyp)} % The in-sample test error.
\newcommand{\EPE}{\risk(\hat{\hyp}_n)} % The out-of-sample test error.
\newcommand{\folds}{K} % Cross validation folds 
\newcommand{\fold}{k} % Index of a fold
\newcommand{\bootstraps}{B} % Bootstrap samples
\newcommand{\bootstrap}{{b^*}} % Index of a bootstrap replication


\newcommand{\rankings}{\mathcal{R}} % Rankings, for colaborative filtering.
\newcommand{\ranking}{\mathcal{R}} % Rankings, for colaborative filtering.
\newcommand{\KL}[2]{D_{KL}\left(#1 \Vert #2 \right)}
\newcommand{\ortho}{\mathbb{O}} % space of orthogonal matrices

\newcommand{\id}[6]{
	\begin{tabular}{|p{2cm}|p{2cm}|p{2cm}|p{2cm}|p{2cm}|p{2cm}|}
	\hline Task & Type & Input & Output & Concept & Remark \\ 
	\hline 
	\hline #1 & #2 & #3 & #4 & #5 & #6 \\ 
	\hline 
	\end{tabular} 
	\newline
	\newline
}

\newcommand{\union}{\cup}
\newcommand{\intersect}{\cap}
\newcommand{\supp}[1]{\mathop{support}(#1)}
\newcommand{\conf}[2]{\mathop{confidence}(#1 \Rightarrow #2)}
\newcommand{\lift}[2]{\mathop{lift}(#1 \Rightarrow #2)}
\newcommand{\convic}[2]{\mathop{conviction}(#1 \Rightarrow #2)}


\newcommand{\machine}[1]{\estim{\theta}_n^{(#1)}}
\newcommand{\minimizer}{\theta^*}
\newcommand{\generative}{\theta_0}
\newcommand{\parallelized}{\bar{\theta}_{N,m}}
\newcommand{\parallelizedII}{\mathring{\theta}_{N,m}}
\newcommand{\parallelizedIII}{\prescript{}{2}{\widehat{\theta}}_{N,m}}
\newcommand{\centralized}{\estim{\theta}_N}
\newcommand{\parallelKL}{\estim{\theta}_{KL}}
\newcommand{\penalize}{J}
\newcommand{\bigO}{\mathcal{O}}
\newcommand{\bigOprob}{\mathcal{O}_P}
\newcommand{\smallO}{o}
\newcommand{\smallOprob}{o_P}

\newcommand{\citeJR}[1]{\citeauthor{#1} \citep{#1}}
\newcommand{\citeJRfull}[1]{\citeauthor*{#1} \citep{#1}}
\newcommand{\error}{\mathcal{E}}

\newcommand{\M}{$M$}
\newcommand{\MII}{$\prescript{}{2}{M}$}

\newcommand{\biasSecond}[1]{B_2(#1)}
\newcommand{\MSESecond}[1]{M_2(#1)}

\newcommand{\rate}{r}

\newcommand{\emptyfigure}[1]{\missingfigure[figwidth=6cm]{#1}}


% % Time line
%\usepackage[paperwidth=210mm,%
%    paperheight=297mm,%
%    tmargin=7.5mm,%
%    rmargin=7.5mm,%
%    bmargin=7.5mm,%
%    lmargin=7.5mm,
%    vscale=1,%
%    hscale=1]{geometry}
%
%\usepackage[utf8]{inputenc}
%\usepackage[T1]{fontenc}

\usepackage{tikz}
\usetikzlibrary{arrows, calc, decorations.markings, positioning}


\makeatletter
\newenvironment{timeline}[6]{%
    % #1 is startyear
    % #2 is tlendyear
    % #3 is yearcolumnwidth
    % #4 is rulecolumnwidth
    % #5 is entrycolumnwidth
    % #6 is timelineheight

    \newcommand{\startyear}{#1}
    \newcommand{\tlendyear}{#2}

    \newcommand{\yearcolumnwidth}{#3}
    \newcommand{\rulecolumnwidth}{#4}
    \newcommand{\entrycolumnwidth}{#5}
    \newcommand{\timelineheight}{#6}

    \newcommand{\templength}{}

    \newcommand{\entrycounter}{0}

    % http://tex.stackexchange.com/questions/85528/checking-whether-or-not-a-node-has-been-previously-defined
    % http://tex.stackexchange.com/questions/37709/how-can-i-know-if-a-node-is-already-defined
    \long\def\ifnodedefined##1##2##3{%
        \@ifundefined{pgf@sh@ns@##1}{##3}{##2}%
    }

    \newcommand{\ifnodeundefined}[2]{%
        \ifnodedefined{##1}{}{##2}
    }

    \newcommand{\drawtimeline}{%
        \draw[timelinerule] (\yearcolumnwidth+5pt, 0pt) -- (\yearcolumnwidth+5pt, -\timelineheight);
        \draw (\yearcolumnwidth+0pt, -10pt) -- (\yearcolumnwidth+10pt, -10pt);
        \draw (\yearcolumnwidth+0pt, -\timelineheight+15pt) -- (\yearcolumnwidth+10pt, -\timelineheight+15pt);

        \pgfmathsetlengthmacro{\templength}{neg(add(multiply(subtract(\startyear, \startyear), divide(subtract(\timelineheight, 25), subtract(\tlendyear, \startyear))), 10))}
        \node[year] (year-\startyear) at (\yearcolumnwidth, \templength) {\startyear};

        \pgfmathsetlengthmacro{\templength}{neg(add(multiply(subtract(\tlendyear, \startyear), divide(subtract(\timelineheight, 25), subtract(\tlendyear, \startyear))), 10))}
        \node[year] (year-\tlendyear) at (\yearcolumnwidth, \templength) {\tlendyear};
    }

    \newcommand{\entry}[2]{%
        % #1 is the year
        % #2 is the entry text

        \pgfmathtruncatemacro{\lastentrycount}{\entrycounter}
        \pgfmathtruncatemacro{\entrycounter}{\entrycounter + 1}

        \ifdim \lastentrycount pt > 0 pt%
            \node[entry] (entry-\entrycounter) [below of=entry-\lastentrycount] {##2};
        \else%
            \pgfmathsetlengthmacro{\templength}{neg(add(multiply(subtract(\startyear, \startyear), divide(subtract(\timelineheight, 25), subtract(\tlendyear, \startyear))), 10))}
            \node[entry] (entry-\entrycounter) at (\yearcolumnwidth+\rulecolumnwidth+10pt, \templength) {##2};
        \fi

        \ifnodeundefined{year-##1}{%
            \pgfmathsetlengthmacro{\templength}{neg(add(multiply(subtract(##1, \startyear), divide(subtract(\timelineheight, 25), subtract(\tlendyear, \startyear))), 10))}
            \draw (\yearcolumnwidth+2.5pt, \templength) -- (\yearcolumnwidth+7.5pt, \templength);
            \node[year] (year-##1) at (\yearcolumnwidth, \templength) {##1};
        }

        \draw ($(year-##1.east)+(2.5pt, 0pt)$) -- ($(year-##1.east)+(7.5pt, 0pt)$) -- ($(entry-\entrycounter.west)-(5pt,0)$) -- (entry-\entrycounter.west);
    }

    \newcommand{\plainentry}[2]{% plainentry won't print date in the timeline
        % #1 is the year
        % #2 is the entry text

        \pgfmathtruncatemacro{\lastentrycount}{\entrycounter}
        \pgfmathtruncatemacro{\entrycounter}{\entrycounter + 1}

        \ifdim \lastentrycount pt > 0 pt%
            \node[entry] (entry-\entrycounter) [below of=entry-\lastentrycount] {##2};
        \else%
            \pgfmathsetlengthmacro{\templength}{neg(add(multiply(subtract(\startyear, \startyear), divide(subtract(\timelineheight, 25), subtract(\tlendyear, \startyear))), 10))}
            \node[entry] (entry-\entrycounter) at (\yearcolumnwidth+\rulecolumnwidth+10pt, \templength) {##2};
        \fi

        \ifnodeundefined{invisible-year-##1}{%
            \pgfmathsetlengthmacro{\templength}{neg(add(multiply(subtract(##1, \startyear), divide(subtract(\timelineheight, 25), subtract(\tlendyear, \startyear))), 10))}
            \draw (\yearcolumnwidth+2.5pt, \templength) -- (\yearcolumnwidth+7.5pt, \templength);
            \node[year] (invisible-year-##1) at (\yearcolumnwidth, \templength) {};
        }

        \draw ($(invisible-year-##1.east)+(2.5pt, 0pt)$) -- ($(invisible-year-##1.east)+(7.5pt, 0pt)$) -- ($(entry-\entrycounter.west)-(5pt,0)$) -- (entry-\entrycounter.west);
    }

    \begin{tikzpicture}
        \tikzstyle{entry} = [%
            align=left,%
            text width=\entrycolumnwidth,%
            node distance=10mm,%
            anchor=west]
        \tikzstyle{year} = [anchor=east]
        \tikzstyle{timelinerule} = [%
            draw,%
            decoration={markings, mark=at position 1 with {\arrow[scale=1.5]{latex'}}},%
            postaction={decorate},%
            shorten >=0.4pt]

        \drawtimeline
}
{
    \end{tikzpicture}
    \let\startyear\@undefined
    \let\tlendyear\@undefined
    \let\yearcolumnwidth\@undefined
    \let\rulecolumnwidth\@undefined
    \let\entrycolumnwidth\@undefined
    \let\timelineheight\@undefined
    \let\entrycounter\@undefined
    \let\ifnodedefined\@undefined
    \let\ifnodeundefined\@undefined
    \let\drawtimeline\@undefined
    \let\entry\@undefined
}
\makeatother
% % % % %

\newcommand{\R}{\textnormal{\sffamily\bfseries R }}

% Process capability notation
\newcommand{\targetValue}{T}% target value
\newcommand{\cp}{C_p}% c_p
\newcommand{\cpHat}{\hat{C}_p}% c_p
\newcommand{\ctqExpect}{\mu}
\newcommand{\pnc}{p_{NC}}
\newcommand{\cpu}{C_{pu}}
\newcommand{\cpl}{C_{pl}}
\newcommand{\cpk}{C_{pk}}
\newcommand{\cpm}{C_{pm}}
\newcommand{\cpq}{C_p(q)}
\newcommand{\pp}{P_{p}}
\newcommand{\ppk}{P_{pk}}

\newcommand{\barxChart}{$\bar{x}$-chart}
\newcommand{\sigmabar}{\sigma_{\bar{x}}}
\newcommand{\aka}{{a.k.a.\ }}
\newcommand{\Aka}{{A.k.a.\ }}
\newcommand{\rcode}[1]{\texttt{#1}}
\newcommand{\arm}{L}

\newcommand{\tsq}{$T^2$ }